
\documentclass[11pt,a4paper]{article}

\usepackage[a4paper,margin=2.5cm]{geometry}
\usepackage[brazil]{babel}
\usepackage[T1]{fontenc}
\usepackage[utf8]{inputenc}
\usepackage{setspace}
\usepackage{titlesec}
\usepackage{graphicx}
\usepackage{booktabs}
\usepackage{array}
\usepackage{tabularx}
\usepackage{longtable}
\usepackage{enumitem}
\usepackage{hyperref}

\setstretch{1.2}

\titleformat{\section}{\large\bfseries}{\thesection}{1em}{}
\titleformat{\subsection}{\normalsize\bfseries}{\thesubsection}{1em}{}
\titleformat{\subsubsection}{\normalsize\bfseries}{\thesubsubsection}{1em}{}

\newcolumntype{L}[1]{>{\raggedright\arraybackslash}p{#1}}
\newcolumntype{Y}{>{\raggedright\arraybackslash}X}

\begin{document}

\begin{center}
{\LARGE \textbf{RÚBRICA DE AVALIAÇÃO — DISCIPLINA DE FUNDAÇÕES}}\\[0.2cm]
{\large ML, Programação e BIM (pyRevit) como ferramentas de apoio didático}\\[0.2cm]
Curso de Engenharia Civil — FECFAU / UNICAMP\\
Professor Responsável: Prof.\ Dr.\ Paulo Albuquerque
\end{center}

\vspace{0.5cm}

\section*{Princípio Geral de Avaliação}

O domínio do raciocínio clássico de engenharia é condição necessária para aprovação. 
O uso de ML, BIM e Digital Twins não compensa erros conceituais de projeto.

\section*{Projeto 1 — Fundações Rasas}

\subsection*{Pesos}
\begin{center}
\begin{tabular}{l c}
\toprule
Item & Peso \\
\midrule
Dimensionamento clássico & 50\% \\
Projeto gráfico & 20\% \\
Machine Learning & 20\% \\
Qualidade do relatório & 10\% \\
\bottomrule
\end{tabular}
\end{center}

\subsection*{1. Dimensionamento Clássico (50\%) — Rubrica detalhada}

\begin{longtable}{L{4.2cm} L{4.2cm} L{4.2cm} L{3.2cm}}
\toprule
\textbf{Critério} & \textbf{Excelente} & \textbf{Adequado} & \textbf{Insuficiente} \\
\midrule
\endhead
Interpretação do SPT & Correta, coerente e bem justificada & Pequenas simplificações & Erros conceituais \\
Cálculo da tensão admissível & Método adequado e bem aplicado & Aplicação parcial & Método incorreto \\
Dimensionamento das sapatas & Coerente, seguro e normativo & Pequenas inconsistências & Dimensionamento incorreto \\
Uso das normas & Citações corretas (NBR~6122) & Uso superficial & Ausente ou incorreto \\
\bottomrule
\end{longtable}

\noindent \textbf{Observação:} erro grave nesta etapa limita a nota máxima do projeto.

\subsection*{2. Projeto Gráfico e Coerência BIM (20\%)}

\begin{center}
\begin{tabularx}{\textwidth}{L{5.2cm} Y}
\toprule
\textbf{Critério} & \textbf{Avaliação} \\
\midrule
Planta de fundações & Clareza, cotas e simbologia \\
Cortes e detalhes & Correção técnica \\
Compatibilidade com o cálculo & Consistência \\
Uso introdutório do pyRevit & Organização e verificação de parâmetros \\
\bottomrule
\end{tabularx}
\end{center}

\subsection*{3. Machine Learning (20\%)}

\noindent
Nesta etapa, o \textbf{Machine Learning é empregado exclusivamente como ferramenta de
alfabetização conceitual}, por meio do uso do software \textbf{Orange (ambiente low-code)}.
O objetivo é desenvolver a compreensão dos elementos fundamentais de um problema de ML
(\textit{features}, variável-alvo, erro, validação e limitações do modelo), 
\textbf{sem que o ML atue como decisor de projeto}.

\begin{center}
\begin{tabularx}{\textwidth}{L{6.2cm} Y}
\toprule
\textbf{Critério} & \textbf{Avaliação} \\
\midrule
Uso correto do dataset & Sim / Não \\
Escolha do modelo & Justificada \\
Métricas adequadas & MAE, erro percentual \\
Comparação com cálculo clássico & Discussão técnica \\
Consciência dos limites do ML & Evidente \\
\bottomrule
\end{tabularx}
\end{center}


\noindent \textbf{Observação:} o uso do ML possui caráter formativo.
Modelos complexos ou automatizações que substituam o raciocínio de engenharia
não geram nota adicional.

\subsection*{4. Qualidade do Relatório (10\%)}

\begin{itemize}[leftmargin=1.2cm]
\item Clareza técnica;
\item Organização;
\item Coerência entre texto, tabelas e resultados.
\end{itemize}

\section*{Projeto 2 — Fundações Profundas}

\subsection*{Pesos}
\begin{center}
\begin{tabular}{l c}
\toprule
Item & Peso \\
\midrule
Dimensionamento clássico & 40\% \\
Análise no RSPile & 20\% \\
Machine Learning & 20\% \\
BIM e uso do pyRevit & 10\% \\
Qualidade técnica e crítica & 10\% \\
\bottomrule
\end{tabular}
\end{center}

\subsection*{1. Dimensionamento Clássico (40\%) — Rubrica detalhada}

\begin{longtable}{L{4.2cm} L{4.2cm} L{4.2cm} L{3.2cm}}
\toprule
\textbf{Critério} & \textbf{Excelente} & \textbf{Adequado} & \textbf{Insuficiente} \\
\midrule
\endhead
Método escolhido & Justificado e correto & Justificado parcialmente & Inadequado \\
Cálculo da capacidade & Correto e consistente & Pequenos desvios & Erros conceituais \\
Definição de comprimento & Bem fundamentada & Simplificada & Arbitrária \\
\bottomrule
\end{longtable}

\subsection*{2. Análise no RSPile (20\%)}

\begin{center}
\begin{tabularx}{\textwidth}{L{6.2cm} Y}
\toprule
\textbf{Critério} & \textbf{Avaliação} \\
\midrule
Modelagem correta & Sim / Não \\
Parâmetros coerentes & Sim / Não \\
Comparação com cálculo analítico & Discussão clara \\
\bottomrule
\end{tabularx}
\end{center}

\subsection*{3. Machine Learning (20\%)}

\noindent
O uso do ML neste projeto permanece subordinado ao dimensionamento clássico,
tendo como finalidade a interpretação crítica de resultados e análise de incertezas.

\begin{center}
\begin{tabularx}{\textwidth}{L{6.2cm} Y}
\toprule
\textbf{Critério} & \textbf{Avaliação} \\
\midrule
Definição clara do problema & Sim / Não \\
Escolha de apenas uma saída & Obrigatório \\
Limite de modelos respeitado & Até 2 \\
Interpretação física dos resultados & Clara \\
Discussão de erros e incertezas & Presente \\
\bottomrule
\end{tabularx}
\end{center}

\noindent \textbf{Observação:} treinar ML com o próprio dado do projeto invalida esta etapa.

\subsection*{4. BIM e uso do pyRevit (10\%)}

\begin{center}
\begin{tabularx}{\textwidth}{L{6.2cm} Y}
\toprule
\textbf{Critério} & \textbf{Avaliação} \\
\midrule
Modelo BIM coerente & Sim / Não \\
Parâmetros corretamente atribuídos & Sim / Não \\
Uso do pyRevit para verificação e rastreabilidade & Funcional e justificado \\
Coerência entre dados, hipóteses e modelo & Evidente \\
\bottomrule
\end{tabularx}
\end{center}


\subsection*{5. Qualidade Técnica e Crítica (10\%)}

\begin{itemize}[leftmargin=1.2cm]
\item Capacidade de argumentação;
\item Reconhecimento de limitações;
\item Postura profissional.
\end{itemize}

\section*{Critérios Transversais (ambos os projetos)}

\subsection*{Penalizações}

\begin{itemize}[leftmargin=1.2cm]
\item Uso de ML como ``decisor'' $\rightarrow$ penalização direta;
\item Falta de memorial de cálculo $\rightarrow$ reprovação do projeto;
\item Copiar código sem entendimento $\rightarrow$ penalização.
\end{itemize}

\subsection*{Bonificação (limitada)}

\begin{itemize}[leftmargin=1.2cm]
\item Discussão crítica madura;
\item Clareza excepcional;
\item Organização exemplar.
\end{itemize}

\noindent \textbf{Observação:} não há bônus por complexidade algorítmica.

\section*{Frase-chave para os alunos (Moodle)}

\begin{quote}
``Nesta disciplina, tecnologia não substitui engenharia.
Um projeto simples, correto e bem justificado vale mais do que um projeto sofisticado sem base técnica.''
\end{quote}

\end{document}