\documentclass[11pt,a4paper]{article}

\usepackage[a4paper,margin=2.5cm]{geometry}
\usepackage[brazil]{babel}
\usepackage[T1]{fontenc}
\usepackage[utf8]{inputenc}
\usepackage{setspace}
\usepackage{booktabs}
\usepackage{enumitem}
\usepackage{hyperref}

\setstretch{1.2}

\begin{document}

\begin{center}
{\LARGE \textbf{CRONOGRAMA DIDÁTICO — TRILHA TECNOLÓGICA}}\\[0.2cm]
{\large Disciplina de Fundações — Engenharia Civil}\\
FECFAU / UNICAMP
\end{center}

\vspace{0.4cm}

\section*{Premissas Gerais}

\begin{itemize}[leftmargin=1.2cm]
\item O roteiro clássico da disciplina de Fundações é conduzido pelo professor responsável, conforme metodologia tradicional.
\item A presente trilha tecnológica atua de forma complementar, sem substituir o ensino dos métodos clássicos.
\item A introdução de ferramentas computacionais ocorre de forma incremental, com controle explícito da sobrecarga cognitiva.
\item O raciocínio de engenharia precede qualquer aplicação de tecnologia.
\item O desempenho computacional não se sobrepõe à coerência técnica e normativa.
\end{itemize}

\section*{Visão Geral do Semestre}

\begin{center}
\begin{tabular}{l c c}
\toprule
Fase & Semanas & Foco Principal \\
\midrule
Fase 0 & 1 & Alinhamento conceitual e papel da tecnologia \\
Fase 1 & 2--6 & Projeto 1: ML introdutório (Orange) e BIM básico \\
Fase 2 & 7--8 & Consolidação e entrega do Projeto 1 \\
Fase 3 & 9--13 & Projeto 2: ML moderado (Python) e BIM aplicado \\
Fase 4 & 14--15 & Consolidação e entrega do Projeto 2 \\
\bottomrule
\end{tabular}
\end{center}

\section*{Fase 0 — Alinhamento Conceitual}

\textbf{Semana 1}

Objetivos:
\begin{itemize}[leftmargin=1.2cm]
\item Estabelecer limites e responsabilidades do uso de tecnologia na Engenharia Geotécnica;
\item Reduzir resistência inicial ao uso de ferramentas computacionais;
\item Alinhar expectativas quanto aos projetos e critérios de avaliação.
\end{itemize}

Conteúdo:
\begin{itemize}[leftmargin=1.2cm]
\item O que Machine Learning \textbf{não} é em engenharia;
\item Integração entre dados, modelos computacionais e representação BIM aplicada a fundações;
\item Apresentação geral dos projetos, datasets e ferramentas.
\end{itemize}

\section*{Fase 1 — Projeto 1: ML Introdutório e BIM Simples}

\textbf{Semanas 2 a 6}

\subsection*{Semana 2 — Dados e Engenharia}

Conteúdo:
\begin{itemize}[leftmargin=1.2cm]
\item Conceito de dataset e variáveis com significado físico;
\item Apresentação do dataset-base de fundações rasas.
\end{itemize}

Atividade:
\begin{itemize}[leftmargin=1.2cm]
\item Análise exploratória dos dados (tabelas e gráficos);
\item Discussão da relação entre dados e hipóteses de engenharia.
\end{itemize}

\subsection*{Semana 3 — Machine Learning Introdutório (Orange)}

Conteúdo:
\begin{itemize}[leftmargin=1.2cm]
\item Aprendizagem supervisionada e regressão linear;
\item Métricas de erro e interpretação física.
\end{itemize}

Atividade:
\begin{itemize}[leftmargin=1.2cm]
\item Execução do modelo obrigatório em ambiente visual (Orange);
\item Comparação preliminar com o dimensionamento clássico.
\end{itemize}

\subsection*{Semana 4 — Limites do Machine Learning}

Conteúdo:
\begin{itemize}[leftmargin=1.2cm]
\item Overfitting, escassez de dados e extrapolação;
\item Discussão crítica dos resultados obtidos via ML.
\end{itemize}

Atividade:
\begin{itemize}[leftmargin=1.2cm]
\item Relatório curto sobre limitações e falhas do modelo.
\end{itemize}

\subsection*{Semana 5 — BIM Básico}

Conteúdo:
\begin{itemize}[leftmargin=1.2cm]
\item Modelagem de fundações rasas no Revit;
\item Parâmetros geométricos e introdução ao pyRevit (leitura e organização).
\end{itemize}

Atividade:
\begin{itemize}[leftmargin=1.2cm]
\item Modelo BIM simples das sapatas.
\end{itemize}

\subsection*{Semana 6 — Integração do Projeto 1}

\begin{itemize}[leftmargin=1.2cm]
\item Integração entre cálculo clássico, ML introdutório e BIM;
\item Preparação da entrega do Projeto 1.
\end{itemize}

\section*{Fase 2 — Consolidação do Projeto 1}

\textbf{Semanas 7 e 8}

\begin{itemize}[leftmargin=1.2cm]
\item Entrega do Projeto 1;
\item Discussão coletiva dos resultados;
\item Consolidação conceitual do papel do ML como ferramenta de apoio.
\end{itemize}

\section*{Fase 3 — Projeto 2: ML Moderado e BIM Aplicado}

\textbf{Semanas 9 a 13}

\subsection*{Semana 9 — Dados Estratificados e Definição do Problema}

\begin{itemize}[leftmargin=1.2cm]
\item Datasets geotécnicos estratificados;
\item Definição clara da variável de saída do ML.
\end{itemize}

\subsection*{Semana 10 — Machine Learning Moderado (Python)}

\begin{itemize}[leftmargin=1.2cm]
\item Uso de notebooks-base fornecidos;
\item Comparação entre até dois modelos de ML;
\item Ênfase em reprodutibilidade.
\end{itemize}

\subsection*{Semana 11 — Integração com RSPile}

\begin{itemize}[leftmargin=1.2cm]
\item Comparação entre ML, cálculo analítico e RSPile;
\item Discussão técnica de divergências.
\end{itemize}

\subsection*{Semana 12 — BIM Aplicado e Rastreamento de Dados}

\begin{itemize}[leftmargin=1.2cm]
\item Modelagem BIM de tubulões e elementos associados;
\item Uso do pyRevit para verificação, rastreabilidade e extração de dados;
\item Integração entre modelo, hipóteses adotadas e memorial de cálculo.
\end{itemize}

\subsection*{Semana 13 — Integração Final}

\begin{itemize}[leftmargin=1.2cm]
\item Projeto como sistema integrado de dados, modelos e hipóteses;
\item Discussão sobre limites da automação.
\end{itemize}

\section*{Fase 4 — Encerramento}

\textbf{Semanas 14 e 15}

\begin{itemize}[leftmargin=1.2cm]
\item Entrega do Projeto 2;
\item Discussão crítica sobre automação, ética e responsabilidade técnica;
\item Encerramento da trilha tecnológica.
\end{itemize}

\section*{Mensagem Final}

\begin{quote}
A trilha tecnológica acompanha o avanço do conteúdo clássico da disciplina, sendo introduzida de forma incremental e subordinada ao raciocínio tradicional da Engenharia Geotécnica, priorizando compreensão conceitual, julgamento profissional e responsabilidade técnica.
\end{quote}

\end{document}