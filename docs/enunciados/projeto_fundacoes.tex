
\documentclass[11pt,a4paper]{article}

\usepackage[a4paper,margin=2.5cm]{geometry}
\usepackage[brazil]{babel}
\usepackage[T1]{fontenc}
\usepackage[utf8]{inputenc}
\usepackage{setspace}
\usepackage{titlesec}
\usepackage{graphicx}
\usepackage{booktabs}
\usepackage{enumitem}
\usepackage{hyperref}

\setstretch{1.2}

\titleformat{\section}{\large\bfseries}{\thesection}{1em}{}
\titleformat{\subsection}{\normalsize\bfseries}{\thesubsection}{1em}{}

\begin{document}

\begin{center}
{\LARGE \textbf{DISCIPLINA: FUNDAÇÕES}}\\[0.3cm]
Curso de Engenharia Civil -- FECFAU / UNICAMP\\
Professor Responsável: Prof.\ Dr.\ Paulo Albuquerque\\
\end{center}

\vspace{0.5cm}

\section*{Introdução}

A presente disciplina tem como objetivo o ensino de fundações rasas e profundas, com base em métodos clássicos da Engenharia Geotécnica, integrando de forma incremental ferramentas computacionais contemporâneas, como Machine Learning (ML), Modelagem da Informação da Construção (BIM), Digital Twins e scripts em pyRevit.  

O princípio norteador da disciplina é o de que a tecnologia atua como ferramenta de apoio à engenharia, não substituindo o raciocínio técnico, a análise normativa ou a responsabilidade profissional.

\section*{PROJETO 1 --- FUNDAÇÕES RASAS}

\subsection*{Objetivo}

Capacitar o aluno a dimensionar fundações rasas para edificações de pequeno porte, a partir de dados de sondagem SPT, utilizando métodos clássicos consagrados, complementados por uma aplicação introdutória de Machine Learning.

\subsection*{Descrição do Problema}

Cada grupo receberá planta arquitetônica simplificada, cargas verticais atuantes e dados de sondagem SPT do terreno. O projeto deverá contemplar sapatas isoladas, associadas e de divisa, quando aplicável.

\subsection*{Atividades Obrigatórias}

\subsubsection*{Dimensionamento Clássico}

O grupo deverá realizar a interpretação geotécnica do solo, determinação da tensão admissível, dimensionamento das áreas das sapatas e verificações normativas conforme a NBR~6122. Este item é soberano no projeto.

\subsubsection*{Aplicação de Machine Learning}

Será fornecido um dataset-base da disciplina. Cada grupo deverá utilizar obrigatoriamente um modelo de regressão linear e, opcionalmente, um segundo modelo simples. Os resultados obtidos via ML deverão ser comparados criticamente com o dimensionamento clássico.

\subsubsection*{Modelagem BIM}

As fundações deverão ser modeladas no Revit, com atribuição de parâmetros geométricos e de carga. O uso de pyRevit será restrito à leitura e organização de parâmetros.

\subsection*{Entregáveis}

\begin{itemize}[leftmargin=1.2cm]
\item Memorial de cálculo (PDF);
\item Relatório técnico de ML (PDF);
\item Projeto gráfico de fundações;
\item Modelo BIM (.rvt).
\end{itemize}

\subsection*{Critérios de Avaliação}

\begin{center}
\begin{tabular}{l c}
\toprule
Item & Peso \\
\midrule
Dimensionamento clássico & 50\% \\
Projeto gráfico & 20\% \\
Aplicação de ML & 20\% \\
Qualidade do relatório & 10\% \\
\bottomrule
\end{tabular}
\end{center}

\section*{PROJETO 2 --- FUNDAÇÕES PROFUNDAS}

\subsection*{Objetivo}

Desenvolver um projeto completo de fundações profundas, integrando dimensionamento clássico, análises no software RSPile, Machine Learning em nível intermediário e conceitos de Digital Twin.

\subsection*{Descrição do Problema}

Cada grupo receberá dados de sondagem SPT, cargas estruturais e um tipo específico de estaca definido pelo professor, mantendo a prática tradicional da disciplina.

\subsection*{Atividades Obrigatórias}

\subsubsection*{Dimensionamento Clássico}

Determinação da capacidade de carga, comprimento e diâmetro das estacas, conforme métodos analíticos adequados.

\subsubsection*{Análise Numérica}

Modelagem e análise das estacas no software RSPile, com comparação crítica entre resultados analíticos e numéricos.

\subsubsection*{Aplicação de Machine Learning}

Cada grupo deverá escolher apenas uma variável de saída (capacidade última ou comprimento da estaca), utilizando no máximo dois modelos de ML para estimativas preliminares.

\subsubsection*{Digital Twin e BIM}

Modelagem das estacas e blocos no Revit, com uso de pyRevit para leitura de parâmetros, verificação de coerência geométrica e extração de informações do modelo, caracterizando um Digital Twin simplificado.

\subsection*{Entregáveis}

\begin{itemize}[leftmargin=1.2cm]
\item Memorial de cálculo;
\item Relatório técnico integrado (ML + RSPile + BIM);
\item Projeto gráfico;
\item Modelo BIM;
\item Arquivos computacionais utilizados.
\end{itemize}

\subsection*{Critérios de Avaliação}

\begin{center}
\begin{tabular}{l c}
\toprule
Item & Peso \\
\midrule
Dimensionamento clássico & 40\% \\
Análise no RSPile & 20\% \\
Aplicação de ML & 20\% \\
Digital Twin (BIM + pyRevit) & 10\% \\
Qualidade técnica e crítica & 10\% \\
\bottomrule
\end{tabular}
\end{center}

\section*{Observação Final}

O uso de ferramentas computacionais não substitui o raciocínio de engenharia. As decisões técnicas devem ser sempre fundamentadas em critérios normativos e no julgamento profissional do engenheiro responsável.

\end{document}